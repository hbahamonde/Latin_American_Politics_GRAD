% LaTeX Curriculum Vitae Template
%
% Copyright (C) 2004-2009 Jason Blevins <jrblevin@sdf.lonestar.org>
% http://jblevins.org/projects/cv-template/
%
% You may use use this document as a template to create your own CV
% and you may redistribute the source code freely. No attribution is
% required in any resulting documents. I do ask that you please leave
% this notice and the above URL in the source code if you choose to
% redistribute this file.

\documentclass[letterpaper]{article}

\usepackage{hyperref}
\usepackage{geometry}

% Comment the following lines to use the default Computer Modern font
% instead of the Palatino font provided by the mathpazo package.
% Remove the 'osf' bit if you don't like the old style figures.
\usepackage[T1]{fontenc}
\usepackage[sc,osf]{mathpazo}

% Set your name here
\def\name{Latin American Politics}

% Replace this with a link to your CV if you like, or set it empty
% (as in \def\footerlink{}) to remove the link in the footer:
\def\footerlink{}
% \href{http://www.hectorbahamonde.com}{www.HectorBahamonde.com}

% The following metadata will show up in the PDF properties
\hypersetup{
  colorlinks = true,
  urlcolor = blue,
  pdfauthor = {\name},
  pdfkeywords = {political science, comparative politics},
  pdftitle = {\name: Syllabus},
  pdfsubject = {Syllabus},
  pdfpagemode = UseNone
}

\geometry{
  body={6.5in, 8.5in},
  left=1.0in,
  top=1.25in
}

% Customize page headers
\pagestyle{myheadings}
\markright{{\tiny \name}}
\thispagestyle{empty}

% Custom section fonts
\usepackage{sectsty}
\sectionfont{\rmfamily\mdseries\Large}
\subsectionfont{\rmfamily\mdseries\itshape\large}

% Don't indent paragraphs.
\setlength\parindent{0em}

% Make lists without bullets
\renewenvironment{itemize}{
  \begin{list}{}{
    \setlength{\leftmargin}{1.5em}
  }
}{
  \end{list}
}

\begin{document}

% Place name at left
%{\huge \name}

% Alternatively, print name centered and bold:
\centerline{\huge \bf \name}

\vspace{0.25in}

\begin{minipage}{0.45\linewidth}
  Rutgers University, New Brunswick \\
  Political Science Department \\
  Hickman Hall \\
  New Brunswick, NJ 08901\\
  \\
  \\

\end{minipage}
\hspace{4cm}\begin{minipage}{0.45\linewidth}
  \begin{tabular}{ll}
{\bf Last updated}: \today. \\
 {\bf Download last version} \href{https://github.com/hbahamonde/Latin_American_Politics_GRAD/raw/master/Bahamonde_Latin_American_Politics_Syllabus_GRAD.pdf}{here}.\\
   {\bf {\color{red}{\scriptsize Not intended as a definitive version}}} %\\
    \\
    \\
    \\
    \\
    \\
  \end{tabular}
\end{minipage}

\vspace{-5mm}
{\bf Instructor}: H\'ector Bahamonde\\
\texttt{e:}\href{mailto:hector.bahamonde@rutgers.edu}{\texttt{hector.bahamonde@rutgers.edu}}\\
\texttt{w:}\href{http://www.hectorbahamonde.com}{\texttt{www.hectorbahamonde.com}}\\
{\bf Location}: Classroom.\\
{\bf Office Hours}: Make an appointment \href{https://calendly.com/bahamonde/officehours}{\texttt{here}}.\\
{\bf Class Website and Materials}: click \href{https://github.com/hbahamonde/Latin_American_Politics_GRAD}{\texttt{here}}.

\subsection*{Overview and Objectives}

This {\bf {\color{blue}graduate-level course}} is intended as an introduction to Latin American politics from a comparative politics view. The papers and chapters will draw from what call `the core' that defines this important area of research. 



\subsection*{Course Learning Objectives}
 
Upon successful completion of this course, you will be able to:

\begin{itemize}
	\item[$\bullet$] Acquire an understanding of the main democratization and development theories in Latin America.
	\item[$\bullet$] Use the comparative method and analysis in the political science literature.
	\item[$\bullet$] Consume `critically' the Latin American politics literature.
	\item[$\bullet$] Produce original research relevant to the subfield.
\end{itemize}



\subsection*{Requirements}

In this course we will cover the key concepts and theoretical debates in a very large sub-field in political science. Students will be expected to complete the required readings each week, attend the seminar, participate in class discussions and take careful notes. You will also be required to serve as a discussant a number TBA of times. Basically, your job will be to comment on your colleagues' reaction papers with specific references to the material. Two papers at the middle and end of the road will also be required. Based on what I see from our weekly discussion, I will provide the topics. Suggestions are welcomed.


\subsection*{Evaluation}


\begin{itemize}
	\item[$\bullet$] {\bf Weakly reaction papers} - do not write a reaction paper when you serve as a discussant: 40 \%.
	\item[$\bullet$] {\bf Two research papers}: 40 \%.
	\item[$\bullet$] {\bf Participation}: 20 \%.
\end{itemize}


\subsection*{Academic Integrity}
In accordance with Rutgers University policy on Academic Integrity, you are expected to fully comply with the school's policies.  Please see this \href{http://academicintegrity.rutgers.edu}{\texttt{link}}.

\subsection*{Policy on Audits}
You can audit my course. However, I expect you to participate and complete all requirements. 

\subsection*{Students with Disabilities}
Students with disabilities who require accommodation should review the following statement from the Office of Disability Services \href{https://ods.rutgers.edu/faculty/syllabus}{\texttt{link}}.

\subsection*{Office Hours}

I have an open-doors policy, feel free to stop by my office at any time. However, you might want to minimize the risks that I am not there. I advice you then to schedule time with me using my automatic scheduler. I think fixed office hours do not work because ... well, they are fixed. I prefer flexibility. Hence, you can see me any day/time that's available during the week. Do not send me a reminder as I will receive an alert: If the time spot is available, I am happy to see you there. Please follow this \href{https://calendly.com/bahamonde/officehours}{\texttt{link}}.


\subsection*{Cell Phones} 

Make sure your cell phones are turned OFF before entering class.


\subsection*{Schedule}

\begin{enumerate}

\item {\bf History of Latin America}
	\begin{itemize}
		\item[$\bullet$] Collier, R.B., and Collier, D., Shaping the Political Arena. Princeton: Princeton University Press, 1991, 3-20, 27-55, 59-68, 93-106, 161-172, 196-201, 271-272, 314-315, 353-367, 403-406, 438- 439, 469, 498-513, 571-573, 639, 692-693, 745-774.
		\item[$\bullet$] Rueschemeyer, D., Stephens, E.H., and Stephens, J.D., Capitalist Development and Democracy. Chicago: University of Chicago Press, 1992, 155-225.
	\end{itemize}



\item {\bf State-Building and Development}
	\begin{itemize}
		%\item[$\bullet$] Valenzuela, J.S., and Valenzuela, A., ``Modernization and Dependency: Alternative Perspectives in the Study of Latin American Underdevelopment,'' Comparative Politics 10 no. 4 (July 1978), pp. 535-552.
		\item[$\bullet$] Sokoloff, K. L. and S. L. Engerman. ``Institutions, Factor Endowments, and Paths of Development in the New World.'' Journal of Economic Perspectives 14, No. 3 (2000): 217-232.
		\item[$\bullet$] Coatsworth, J. ``Inequality, Institutions, and Economic Growth in Latin America.'' Journal of Latin American Studies 40, No. 3 (2008): 545-569.
		\item[$\bullet$] Dell, M. ``The Persistent Effects of Peru's Mining Mita.'' Econometrica 78 (6) (2010): 1863-1903.
		\item[$\bullet$] Mahoney, J. Colonialism and Postcolonial Development: Spanish America in Comparative Perspective (Cambridge University Press, 2010), Chapters 1 and 8.
		%\item[$\bullet$] North, D., W. Summerhill, and B. Weingast, ``Order, Disorder, and Economic Change: Latin America versus North America.'' In B. Bueno de Mesquita and H. Root, eds. Governing for Prosperity (New Haven: Yale University Press, 2000).
		\item[$\bullet$] Centeno, M., Blood and Debt: War and the Nation-State in Latin America (Penn State University Press, 2002), 1-26, 33-47, 261-280.
		\item[$\bullet$] Soifer, H. State Building in Latin America (Cambridge University Press, 2015), chapters 1, 2, and 7.
		\item[$\bullet$] Kurtz, M., Latin American State-Building in Comparative Perspective: Social Foundations of Institutional Order. New York: Cambridge University Press, 2013, Chapters 1-2 (pp. 1-65).
	\end{itemize}


\item {\bf Development Strategies I}
	\begin{itemize}
		\item[$\bullet$] Bulmer-Thomas, V., The Economic History of Latin America since Independence, 3rd edition (Cambridge University Press, 2014), 296-318, 330-345, 383-390, 413-422
		\item[$\bullet$] Prebisch, R., ``International Trade and Payments in an Era of Coexistence: Commercial Policy in the Underdeveloped Countries,'' The American Economic Review 49 no. 2 (May 1959): 251- 273.
		\item[$\bullet$] Hirschman, A.O., ``The Political Economy of Import-Substituting Industrialization in Latin America,'' The Quarterly Journal of Economics 82 no. 1 (February 1968): 1-32.
		\item[$\bullet$] Baer, W., ``Import Substitution and Industrialization in Latin America: Experiences and Interpretations,'' Latin American Research Review 7 no. 1 (Spring 1972): 95-111.
		%\item[$\bullet$] Dominguez J. I., ``Explaining Latin America's Lagging Development in the Second Half of the Twentieth Century: Growth Strategies, Inequality, and Economic Crises,'' in Falling Behind: Explaining the Development Gap between Latin America and the United States, ed. F. Fukuyama (Oxford University Press, 2008), 72-96.
		\item[$\bullet$] Dornbusch, R., and Edwards, S., ``Macroeconomic Populism,'' Journal of Development Economics 32 (1990): 247-275.
		%\item[$\bullet$] Corbo, V., de Melo, J., and Tybout, J., ``What Went Wrong with the Recent Reforms in the Southern Cone,'' Economic Development and Cultural Change 34 no. 3 (April 1986): 607-637.
		\item[$\bullet$] Bahamonde, H. ``Structural Transformations and State Institutions in Latin America, 1900-2010.'' 2016.
		\item[$\bullet$] Blejer, M.I., and Cheasty, A., ``High Inflation, Heterodox Stabilization, and Fiscal Policy,'' World Development 16 no. 8 (August 1988): 867-879.
		\item[$\bullet$] Pastor, M., ``Bolivia: Hyperinflation, Stabilization, and Beyond,'' Journal of Development Studies 27 no. 2 (January 1991): 211-233.
		\item[$\bullet$] Roxborough, I., ``Inflation and Social Pacts in Brazil and Mexico,'' Journal of Latin American Studies 24 (October 1992): 639-664.
	\end{itemize}



\item {\bf Development Strategies II}
	\begin{itemize}
		\item[$\bullet$] Haggard, S., and Kaufman, R.R., The Political Economy of Democratic Transitions. Princeton University Press, 1995, pp. 3-20, 151-218, 227.
		\item[$\bullet$] Geddes, B., ``The Politics of Economic Liberalization,'' Latin American Research Review 30 no. 2 (1995): 195-214.
		\item[$\bullet$] Schamis, H. ``Distributional Coalitions and the Politics of Economic Reform in Latin America.'' World Politics 51, No. 2 (1999): 236-268.
		%\item[$\bullet$] Baker, A. ``Why is Trade Reform so Popular in Latin America? A Consumption-Based Theory of Trade Policy Preferences.'' World Politics 55, No. 3 (April 2003): 423-455.
		\item[$\bullet$] Campello, D. The Politics of Market Discipline in Latin America (New York: Cambridge University Press, 2015), pp. 1-22.
		\item[$\bullet$] Flores-Macias, G. ``Statist vs. Pro-Market: Explaining Leftist Governments' Economic Policies in Latin America.'' Comparative Politics 42, No. 4 (July 2010): 413-433.
		\item[$\bullet$] Weyland, K., ``The Rise of Latin America's Two Lefts: Insights from Rentier State Theory,'' Comparative Politics 41:2 (January 2009): 145-164.
		%\item[$\bullet$] Stallings, B., and Peres, W., ``Is Economic Reform Dead in Latin America? Rhetoric and Reality since 2000,'' Journal of Latin American Studies 43 (2011): 755-786.
		\item[$\bullet$] Haggard, S., and Kaufman, R.R., Development, Democracy and Welfare States: Latin America East Asia, and Eastern Europe (Princeton University Press, 2008), pp. 1-17; 27-51; 59-65; 71-78; 181-220; 262-304.
		%\item[$\bullet$] Garay, C. Social Policy Expansion in Latin America (New York: Cambridge University Press, Forthcoming), pp. 8-34; 44-116.
		\item[$\bullet$] De la O, A. Crafting Policies to End Poverty in Latin America: The Quiet Transformation. (New York: Cambridge University Press, 2015). pp. 1-23; skim pp. 24-43 and 57-70.
	\end{itemize}


\item {\bf Democratic Instability and Authoritarianism}
	\begin{itemize}
		\item[$\bullet$] Lipset, S.M., Political Man. Garden City, NY: Anchor Books, 1963, pp. 27-62.
		\item[$\bullet$] O'Donnell, G., Modernization and Bureaucratic-Authoritarianism. Institute of International Studies, University of California-Berkeley, 1973, pp. 53-114.
		\item[$\bullet$] Collier, D., ``The Bureaucratic-Authoritarian Model: Synthesis and Priorities for Future Research,'' in David Collier, ed. The New Authoritarianism in Latin America. Princeton: Princeton University Press, 1979, pp. 362-395.
		\item[$\bullet$] Schamis, H.E., ``Reconceptualizing Latin American Authoritarianism in the 1970s: From Bureaucratic Authoritarianism to Neoconservatism,'' Comparative Politics 23 no. 2 (January 1991), pp. 201-216.
		%\item[$\bullet$] Stepan, A., ``The New Professionalism of Internal Warfare and Military Role Expansion,'' in A.F. Lowenthal and J.S. Fitch, Armies and Politics in Latin America. Revised edition. New York: Holmes and Meier, 1986, pp. 134-147.
		%\item[$\bullet$] Stepan, A. ``Political Leadership and Regime Breakdown: Brazil,'' In Juan J. Linz and Alfred Stepan, eds. The Breakdown of Democratic Regimes: Latin America. Johns Hopkins University Press, 1978, pp. 110-137.
		\item[$\bullet$] Magaloni, B., Voting for Autocracy: Hegemonic Party Survival and its Demise in Mexico (New York: Cambridge University Press), pp. 1-28, 44-81.
		\item[$\bullet$] Greene, K., Why Dominant Parties Lose: Mexico's Democratization in Comparative Perspective (New York: Cambridge University Press), pp. 33-64, 71-115.
		\item[$\bullet$] Dunning, T., Crude Democracy: Natural Resource Wealth and Political Regimes (New York: Cambridge University Press), pp. 1-25 and 152-209.
		\item[$\bullet$] Albertus, M. Autocracy and Redistribution: The Politics of Land Reform (New York: Cambridge University Press, 2015), pp. 1-103.
	\end{itemize}



\item {\bf Contemporary Democracy, New Challenges}
	\begin{itemize}
		\item[$\bullet$] O'Donnell, G., and Schmitter, P.C., Transitions from Authoritarian Rule: Tentative Conclusions about Uncertain Democracies. Baltimore: The Johns Hopkins Press, 1986, pp. 3-72.
		\item[$\bullet$] Hagopian, F., ``'Democracy by Undemocratic Means'? Elites, Political Pacts, and Regime Transition in Brazil,'' Comparative Political Studies 23 no. 2 (July 1990), pp. 147-166.
		\item[$\bullet$] Haggard, S., and Kaufman, R., ``The Political Economy of Democratic Transitions.'' Comparative Politics 29, No.3 (April 1997): 285-303.		
		\item[$\bullet$] Mainwaring, S. and A. Perez-Linan. 2014. Democracies and Dictatorships in Latin America: Emergence, Survival. New York: Cambridge University Press, chapters 1, 2, and 4.
		\item[$\bullet$] Stepan, A., Rethinking Military Politics. Princeton: Princeton University Press, 1988, pp. 68-127.
		\item[$\bullet$] Hunter, W., Eroding Military Influence in Brazil: Politicians Against Soldiers (Chapel Hill: University of North Carolina Press, 1997), pp. 5-25 and 139-173.
		%\item[$\bullet$] Pion-Berlin, D. and Trinkunas, H., ``Civilian Praetorianism and Military Shirking during Constitutional Crises in Latin America,'' Comparative Politics 42, No. 4 (July 2010): 395-411.
		\item[$\bullet$] O'Donnell, G., ``Delegative Democracy,'' Journal of Democracy 5 no. 1 (1994), pp. 55-69.
		\item[$\bullet$] Levitsky, S. and J. Loxton, ``Populism and Competitive Authoritarianism in the Andes.'' Democratization 20, No. 1 (2013): 107-136.
		\item[$\bullet$] Mazzuca, S. ``The Rise of Rentier Populism,'' Journal of Democracy 24, No. 2 (April 2013): 108-122.
	\end{itemize}


\item {\bf Party Politics in Latin America I}
	\begin{itemize}
		\item[$\bullet$] Dix, R.H., ``Cleavage Structures and Party Systems in Latin America,'' Comparative Politics 22, No. 1 (October 1989): 23-37.
		\item[$\bullet$] Mainwaring, S. and Scully, T., ``Introduction: Party Systems in Latin America.'' In Mainwaring and Scully, eds., Building Democratic Institutions: Party Systems in Latin America. Stanford University Press, 1995.
		%\item[$\bullet$] Samuels, D. and Matthew S. Shugart, Presidents, Parties, and Prime Ministers: How the Separation of Powers Affects Party Organization and Behavior. Cambridge University Press, 2010, pp. 1-18; 22-55; 193-217.
		%\item[$\bullet$] Mainwaring, S., and Zoco, E., ``Political Sequences and the Stabilization of Interparty Competition,'' Party Politics 13, No. 2 (2007): 155-178.
		\item[$\bullet$] Roberts, K. and Wibbels, E., ``Party Systems and Electoral Volatility in Latin America: A Test of Economic, Institutional, and Structural Explanations.'' American Political Science Review 93, No. 3 (September 1999), pp. 575-590.
		\item[$\bullet$] Lupu, N. ``Brand Dilution and the Breakdown of Political Parties in Latin America.'' World Politics 66, No. 4 (October 2014): 561-602.
		\item[$\bullet$] Roberts, K. ``Market Reform, Programmatic (De) alignment, and Party System Stability in Latin America,'' Comparative Political Studies 46, No. 11 (2013): 1422-52.
		\item[$\bullet$] Hagopian, F., Gervasoni, C., and Moraes, J.A., ``From Patronage to Program: The Emergence of Party-Oriented Legislators in Brazil,'' Comparative Political Studies 42, No. 3 (March 2009), pp. 360-391.
		\item[$\bullet$] Handlin, S. ``Social Protection and the Politicization of Class Cleavages during Latin America's Left Turn,'' Comparative Political Studies 46, No. 12: 1582-1609.
		%\item[$\bullet$] Levitsky, S., J. Loxton, and B. Van Dyck, ``Introduction: Challenges of Party-Building in Latin America.'' In S. Levitsky, J. Loxton, B. Van Dyck, and J. Dominguez eds. Challenges of Party-Building in Latin America (New York: Cambridge University Press, 2016).
	\end{itemize}



\item {\bf Party Politics in Latin America II}
	\begin{itemize}
		%\item[$\bullet$] Levitsky, S., and Roberts, K., eds., The Resurgence of the Latin American Left (The Johns Hopkins University Press, 2011), 1-3 (Levistky and Roberts), 31-51 (D. Samuels and J. R. Arnold).
		%\item[$\bullet$] Murillo, M.V., Oliveros, V., and Vaishnav, M., ``Electoral Revolution or Democratic Alternation?'' Latin American Research Review 45:3 (2010): 87-114.
		%\item[$\bullet$] Baker, A., and Greene, K., ``The Latin American Left's Mandate: Free-Market Policies and Issue Voting in New Democracies,'' World Politics 63:1 (January 2011): 43-77.
		%\item[$\bullet$] Campello, D., and Zucco, C., ``Presidential Success and the World Economy,'' The Journal of Politics 78:2 (2016): 589-602.
		\item[$\bullet$] Stokes, S., Dunning, T., Nazareno, M., and Brusco, V., Brokers, Voters, and Clientelism: The Puzzle of Distributive Politics (Cambridge University Press, 2013), 3-14, 18-21, 31-32, 54-55, 65-68, 72, 96-129.
		\item[$\bullet$] Schaffer, J., and Baker, A., ``Clientelism as Persuasion-Buying: Evidence from Latin America,'' Comparative Political Studies 48:9 (2015): 1093-1126.
		\item[$\bullet$] Bahamonde, H. ``Aiming Right at You: Group vs. Individual Clientelistic Targeting in Brazil.'' 2017.
		\item[$\bullet$] Hidalgo, F. D., and Nichter, S., ``Voter Buying: Shaping the Electorate through Clientelism,'' American Journal of Political Science 60, no. 2 (April 2016): 436-455.
		\item[$\bullet$] Larreguy, H., Marshall, J., and Querubin. ``Parties, Brokers, and Voter Mobilization: How Turnout Buying Depends upon the Party's Capacity to Monitor Brokers,'' American Political Science Review 110:1 (February 2016): 160-179.
		\item[$\bullet$] Holland, A. C., and Palmer-Rubin, B., ``Beyond the Machine: Clientelist Brokers and Interest Organizations in Latin America,'' Comparative Political Studies 48:9 (2015): 1186-1223.
		\item[$\bullet$] Weitz-Shapiro, R., ``What Wins Voters: Why Some Politicians Opt Out of Clientelism,'' American Journal of Political Science 56:3 (July 2012): 568-583.
		\item[$\bullet$] Borges-Sugiyama, N., and Hunter, W., ``Whither Clientelism? Good Governance and Brazil's Bolsa Familia Program,'' Comparative Politics 46:1 (October 2013): 43-62.		
	\end{itemize}
\end{enumerate}


 









%\bibliographystyle{plainnat}
%\bibliography{/Users/hectorbahamonde/RU/Bibliografia_PoliSci/Bahamonde_BibTex2013}

\end{document}